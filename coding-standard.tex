%
% These examples are based on the package documentation:
% http://www.ctan.org/tex-archive/macros/latex/contrib/minted
%
\documentclass{article}

\usepackage[T1]{fontenc}
\usepackage[utf8]{inputenc}
\usepackage{lmodern}
\usepackage{textcomp}
\usepackage{minted}

\begin{document}

\title{Padrão de Codificação para a Linguagem C++}
\author{Gustavo Yudi Bientinezi Matsuzake}
\maketitle

%ambiente pra fazer sempre o mesmo código, se quiser mudar o visual do código no minted, mude aqui!!

%CODE
\newminted[code]{cpp}{linenos=true, mathescape, xleftmargin=1cm}
%---------------------%

%SOBRE ESSE DOCUMENTO------------------------------------------------------

\section{Sobre este documento}

Este \emph{padrão de codificação}\footnote{Do inglês, \emph{coding standard}.} foi inspirado nas normas e padrões de codificação do Google\copyright, do projeto GNU\textcopyleft e algumas preferências pessoais do autor. Essas organizações foram escolhidas porque representam uma parcela do mercado de software importante, e aderir os costumes e trejeitos de cada, pode ser de suma importância para garantir uma boa prática organizacional - uma vez que esses métodos sofreram diversos tipos de testes - e acolhida por um grande número de programadores no mundo todo.

\section{Hello World!}

\begin{code}
int main() {
  printf("hello, world");
  return 0;
}
\end{code}

%\section{Math in Source Code Comments}
%\begin{minted}[mathescape,gobble=2]{csharp}
%  /*
%  Defined as $\pi=\lim_{n\to\infty}\frac{P_n}{d}$ where $P$ is the perimeter
%  of an $n$-sided regular polygon circumscribing a
%  circle of diameter $d$.
%  */
%  const double pi = 3.1415926535
%\end{minted}

\end{document}
