%
% These examples are based on the package documentation:
% http://www.ctan.org/tex-archive/macros/latex/contrib/minted
%
\documentclass{article}

\usepackage[T1]{fontenc}
\usepackage[utf8]{inputenc}
\usepackage{lmodern}
\usepackage{textcomp}

%códigos
\usepackage{minted}

\begin{document}

\title{Padrão de Codificação para a Linguagem C++}
\author{Gustavo Yudi Bientinezi Matsuzake}
\maketitle

%ambiente pra fazer sempre o mesmo código, se quiser mudar o visual do código no minted, mude aqui!!

%CODE
\newminted[code]{cpp}{linenos=true, mathescape, xleftmargin=1cm}
%---------------------%

%SOBRE ESSE DOCUMENTO------------------------------------------------------
\section{Sobre este documento}

Este \emph{padrão de codificação}\footnote{Do inglês, \emph{coding standard}.} foi inspirado nas normas e padrões de codificação do Google\copyright, do projeto GNU\textcopyleft e algumas preferências pessoais do autor. Essas organizações foram escolhidas porque representam uma parcela do mercado de software importante, e aderir os costumes e trejeitos de cada, pode ser de suma importância para garantir uma boa prática organizacional - uma vez que esses métodos sofreram diversos tipos de testes - e acolhida por um grande número de programadores no mundo todo.

%DOCUMENTAÇÃO--------------------------------------------------------------
\section{Documentação}

Todo tipo de documentação deve ser mantida atualizada, tanto a da interface gráfica quanto internamente e também a interface de texto, caso sua programação seja de acordo com os padrões \emph{POSIX}\footnote{Conjunto de regras para interfaces de texto e sistemas operacionais inspirado no UNIX.}.

\begin{itemize}
\item Todas as opções de linha de comando (incluindo todos os argumentos --arg) devem ser documentados;
\item Todas as mudanças devem ser documentadas;
\item Em geral, a documentação de todos os aspectos documentados, tanto front-end e back-end devem ser sempre atualizados, e sempre que houver a possibilidade, arrumar e corrigir erros de documentação para evitar gaps de documentação.
\end{itemize}

%ARQUIVOS DE CABEÇÁRIOS-----------------------------------------------------
\section{Arquivos de cebeçário}

Em geral, todo arquivo .cc deve ter um arquivo de cabeçário\footnote{Do inglês, \emph{header}. São os arquivos .h, .hpp ou .H da linguagem c/c++.} .hpp\footnote{Mas nem todo .hpp tem um .cc associoado, como classes com templates totalmente genérico. Mas não recomendamos o uso dessa também, só em algumas exeções.} associoado. Tem algumas exceções, como pequenos .cc e/ou .cc contendo a função main. O uso correto desses arquivos pode fazer a diferença na leitura no tamanho e na performace do código.

%PROTEÇÃO DE DEFINIÇÃO------------------------------------------------------
\section{Proteção contra múltiplos \#includes}

Todo cabeçário deve ser protegido contra  múltiplos \#includes \footnote{Proteção feita por uma definição, para evitar múltiplas definições de estruturas e funções do cabeçário.}. Ou seja, todo cabeçário deve ter a seguinte estrutura ou algo parecido:

\begin{code}
#ifndef _MEU_CABECARIO_H_
#define _MEU_CABECARIO_H_

/*	definicoes	*/
/*	   ...   	*/

#endif \\_MEU_CABECARIO_H_

\end{code}

%OPCOES DO COMPILADOR--------------------------------------------------------
\section{Opções do Compilador}

O compilador deve compilar seu código sem nenhum warning com as tags -Wall -Wextra.

%FUNÇÕES INLINE--------------------------------------------------------------
\section{Funções inline}

Use funções inline apenas quando as funções forem muito pequenas e/ou for pouco reutilizadas. Funções de tamanho significativo pode deixar seu programa mais lento! Seja crítico e pense duas vezes antes de usar funções linearizadas em funções com loops.

%PARAMETROS DE FUNCOES-------------------------------------------------------
\section{Parâmetros de funções}

Quando definir uma função, a ordem dos parâmetros é: entradas, então saídas.

Parâmetros de funções no c/c++ podem ser tanto entradas, quanto saídas (ou os dois!). Entradas são geralmente constantes\footnote{É uma boa prática, ajuda na leitura do código e faz parte da documentação definir entradas de funções como 'const' (constantes).} e saídas são poiteiros. 

\end{document}
