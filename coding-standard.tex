%
% These examples are based on the package documentation:
% http://www.ctan.org/tex-archive/macros/latex/contrib/minted
%
\documentclass{article}

\usepackage[T1]{fontenc}
\usepackage[utf8]{inputenc}
\usepackage{lmodern}
\usepackage{textcomp}
\usepackage{url}
%códigos
\usepackage{minted}
%url
\usepackage[hidelinks]{hyperref}

\begin{document}

\title{Padrão de Codificação para a Linguagem C++}
\author{Gustavo Yudi Bientinezi Matsuzake}
\maketitle

%ambiente pra fazer sempre o mesmo código, se quiser mudar o visual do código no minted, mude aqui!!

%CODE
\newminted[code]{cpp}{linenos=true, mathescape, xleftmargin=1cm}
%---------------------%

%SOBRE ESSE DOCUMENTO------------------------------------------------------
\section{Sobre este documento}

Este\footnote{Disponível em \url{www.github.com/yudi-matsuzake/padrao-de-codificacao-cpp}} \emph{padrão de codificação}\footnote{Do inglês, \emph{coding standard}.} foi inspirado nas normas e padrões de codificação do Google\copyright, do projeto GNU\textcopyleft e algumas preferências pessoais do autor. Essas organizações foram escolhidas porque representam uma parcela do mercado de software importante, e aderir os costumes e trejeitos de cada, pode ser de suma importância para garantir uma boa prática organizacional - uma vez que esses métodos sofreram diversos tipos de testes - e acolhida por um grande número de programadores no mundo todo.

%DOCUMENTAÇÃO--------------------------------------------------------------
\section{Documentação}

Todo tipo de documentação deve ser mantida atualizada, tanto a da interface gráfica quanto internamente e também a interface de texto, caso sua programação seja de acordo com os padrões \emph{POSIX}\footnote{Conjunto de regras para interfaces de texto e sistemas operacionais inspirado no UNIX.}.

\begin{itemize}
\item Todas as opções de linha de comando (incluindo todos os argumentos --arg) devem ser documentados;
\item Todas as mudanças devem ser documentadas;
\item Em geral, a documentação de todos os aspectos documentados, tanto front-end e back-end devem ser sempre atualizados, e sempre que houver a possibilidade, arrumar e corrigir erros de documentação para evitar gaps de documentação.
\end{itemize}

%C++
\section{C++}

C++ é uma linguagem complexa. Tendo isso em mente, primeiramente, seja sensato! Use a linguagens e seus benefícios para o bem, em boas formas. Se você usa a linguagem de forma usual, pense bem antes de usar, pessoas podem fazer a manutenção do seu código, então, priorize a legibilidade e a produtividade. Não estamos em uma competição de quem sabe mais "coisas loucas" sobre a essa linguagem.

%FUNÇÕES INLINE--------------------------------------------------------------
\section{Funções inline}

Use funções inline apenas quando as funções forem muito pequenas e/ou for pouco reutilizadas. Funções de tamanho significativo pode deixar seu programa mais lento! Seja crítico e pense duas vezes antes de usar funções linearizadas em funções com loops.

%DEFINICAO DE CLASSES--------------------------------------------------------
\section{Definição de classes}

Somente defina classes para tipos \emph{não-EDP}. \emph{EDP}\footnote{Os tipos EDP no C++ são definidos como um tipo escalar. Os tipos PDS não tem definidos operadores, destrutores e membros estáticos entre vários tipos do mesmo EDP. Não tem também construtores, membros privados, protegidos ou funções virtuais.} (Estrutura de Dados Passiva) são tipos onde não são necessários as features de orientação a objetos.

%PARAMETROS DE FUNCOES-------------------------------------------------------
\section{Parâmetros de funções}

Quando definir uma função, a ordem dos parâmetros é: entradas, então saídas.

Parâmetros de funções no c/c++ podem ser tanto entradas, quanto saídas (ou os dois!). Entradas são geralmente constantes\footnote{É uma boa prática, ajuda na leitura do código e faz parte da documentação definir entradas de funções como 'const' (constantes).} e saídas são poiteiros. 

%NAMESPACES------------------------------------------------------------------
\section{Namespaces}

Namespaces são encorajados, porque são muito úteis em evitar colisões de nome no escopo global. \emph{inline namespaces} são desencorajados, por outro lado, a não ser em caso de compatibilidade com códigos passados ou softwarares antigos.

inline namespaces são utilizados para casos onde não há diferença entre X::Y::foo() e X::foo() com o código abaixo:

\begin{code}
namespace X{
	inline namespace Y{
		void foo();
	}
}
\end{code}

\subsection{Namespaces sem nome}
Namespaces sem nome são encorajados em arquivos .\{cpp,cc\} mas não faça em .\{h,hpp\}. Em arquivos .\{c,cpp\} namespaces sem nome auxiliam no maior encapsulamento e na proteção de funções e membros desses arquivos, por isso, não faz sentido namespaces sem nome serem utilizados no .\{h,hpp\}.

Exemplo:
\begin{code}
namespace {                           // Isso e um arquivo {cpp,cc}

// O conteudo do namespace nao e identado
//
// Garantido que nao ira gerar simbolos que possam colidir

//Essa funcao nunca ira ser chamada de fora do .cc
bool UpdateInternals(Frobber* f, int newval) {
	/*
	...
	*/
}

}  // namespace
\end{code}

%ARQUIVOS DE CABEÇÁRIOS-----------------------------------------------------
\section{Arquivos de cebeçário}

Em geral, todo arquivo .cc deve ter um arquivo de cabeçário\footnote{Do inglês, \emph{header}. São os arquivos .h, .hpp ou .H da linguagem c/c++.} .hpp\footnote{Mas nem todo .hpp tem um .cc associoado, como classes com templates totalmente genérico. Mas não recomendamos o uso dessa também, só em algumas exeções.} associoado. Tem algumas exceções, como pequenos .cc e/ou .cc contendo a função main. O uso correto desses arquivos pode fazer a diferença na leitura no tamanho e na performace do código.

%PROTEÇÃO DE DEFINIÇÃO------------------------------------------------------
\subsection{Proteção contra múltiplos \#includes}

Todo cabeçário deve ser protegido contra  múltiplos \#includes \footnote{Proteção feita por uma definição, para evitar múltiplas definições de estruturas e funções do cabeçário.}. Ou seja, todo cabeçário deve ter a seguinte estrutura ou algo parecido:

\begin{code}
#ifndef _MEU_CABECARIO_H_
#define _MEU_CABECARIO_H_

/*	definicoes	*/
/*	   ...   	*/

#endif \\_MEU_CABECARIO_H_

\end{code}

\subsection{Nomes e ordem}
Use a seguinte ordem padrão para melhorar a leitura do código e para evitar dependencias escondidas: Cabeçários relacionados, biblioteca padrão do c, biblioteca padrão do c++, outros .\{h,hpp\}, seus .\{h,hpp\}.

Exemplo:
\begin{code}
#include "foo/server/fooserver.h" //relacionado

#include <sys/types.h> //c
#include <unistd.h> //c
#include <hash_map> //c++
#include <vector> //c++

#include "base/basictypes.hpp" //.hpp do projeto
#include "base/commandlineflags.hpp" //.hpp do projeto
#include "foo/server/bar.hpp" //.hpp seu
\end{code}

Todos os cabeçários devem ser listadas decentemente no código do projeto sem o uso de atalhos do padrão UNIX, e.g., .\footnote{Autoreferência} e ..\footnote{Referência do diretório pai}. Por exemplo, o cabeçário life-change-project/src/base/logging.h deverá ser incluído como:

\begin{code}
#include "base/logging.h"
\end{code}

%OPCOES DO COMPILADOR--------------------------------------------------------
\section{Opções do Compilador}

O compilador deve compilar seu código sem nenhum warning com as tags -Wall -Wextra.


%CONVENÇÕES DE FORMATAÇÃO---------------------------------------------------
\section{Convenções de formatação}
\subsection{Tamanho da linha}
É recomendado que a linha possua no máximo 80 caracteres (80 colunas).

\subsection{Nomes}
MACROS devem ser definidas TUDO\_EM\_MAIUSCULA, quando são macros simples. Alguns \emph{wrappers} eficientes para funções, podem ser definidos em letra minúscula. Outros nomes devem ser em letras minúsculas separados\_por\_underline. É comum também, utilizar \_t no final das estruturas, e.g. minha\_estrutura\_t.

\subsection{Expressões}
\begin{tabular}{ l c r }
	{\bf Para} 			& {\bf Use...} 		& {\bf ...ao invés de...} \\
	Negação lógica 			& !x 			& ! x \\
	Complemento bitwise 		& \textasciitilde x 	& \textasciitilde{}  x \\
	Menos unário			& -x			& - x \\
	Cast				& (foo)x		& (foo) x \\
	Referencia de um ponteiro	& *x			& * x \\
\end{tabular}


\subsection{Classes}

Se a definição da classe cabe em uma única linha, deixe em uma única linha. Senão, siga as seguintes regras.

Não indente rótulos de proteção (public, private, virtual).

Prefira colocar o cabeçário da classe em uma única linha.

\begin{code}
class myclass : base {
\end{code}

Se não, coloque o ponto-e-vírgula na outra linha:

\begin{code}
class a_rather_long_class_name 
: with_a_very_long_base_name, and_another_just_to_make_life_hard
{ 
  int member; 
};
\end{code}

Se as cláusulas de herança passar de uma linha, inicialize na próxima linha com a identação de dois espaços:
\begin{code}
class myclass 
: base1 <template_argument1>, base2 <template_argument1>,
  base3 <template_argument1>, base4 <template_argument1>
{ 
  int member; 
};
\end{code}

Quando definir uma classe:

\begin{itemize}
\item Primeiro definir todos os tipos públicos;
\item Então definir todos os não-públicos;
\item Declarar todos os construtores públicos;
\item Declarar todos os destruturoes públicos;
\item Declarar todos os membros de função públicos;
\item Declarar todos os membros de variáveis públicas;
\item Declarar todos os construtores não-públicos;
\item Declarar todos os destrutores não-públicos;
\item Declarar todos os não públicos membros de função e
\item Declarar todos os não públicos membros de variáveis
\end{itemize}



%FIM DO DOCUMENTO--------------------------------------------
\end{document}
